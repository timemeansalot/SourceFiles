% !TeX TS-program = xelatex

\documentclass{resume}
\ResumeName{Fu Jie}

\begin{document}

\ResumeContacts{
  (+86) 151-6213-3916,%
  \ResumeUrl{mailto:timemeansalot@gmail.com}{timemeansalot@gmail.com},%
  \ResumeUrl{https://timemeansalot.github.io/}{Fu Jie's Blog}
  \footnote{The underlined content contains hyperlinks.},%
  \ResumeUrl{https://github.com/timemeansalot}{Github Homepage}%
}

\ResumeTitle

\section{Educational Experience}
\ResumeItem
[CUMT|Master]
{Joint training of ShanghaiTech and ICT}
[\textnormal{Computer Science and Technology}|Master]
[2021.09—2024.06]
\textbf{GPA: 3.02/4.0},main research direction is \textbf{chip design and computer architecture},
\textbf{fresh graduates in 2024}, got the school scholarship twice.

\ResumeItem
[CUMT|Bachelor]
{CUMT}
[\textnormal{Computer Science and Technology} | Bachelor]
[2016.09—2020.06]
\textbf{GPA: 3.4/4.0}: school scholarship once, department scholarship once, scholarships for outstanding freshmen once.

\section{Technical Ability}
\begin{itemize}
  \item \textbf{Programming}: Verilog, Python, C, C++, Chisel; \GrayText{Scala, LaTeX}。
      \footnote{Skills that are not related to the job position are omitted or expressed in gray.}
  \item \textbf{Work Flow}: Linux, Makefile, (Neo)Vim, Git, Tmux.
  \item \textbf{Others}: Familiar with RISC-V ISA, toolchain, processor architecture of some open source cores, English speking and writing.
\end{itemize}

\section{Internship Experience}

\ResumeItem{Institute of Computing Technology, Chinese Academy of Sciences}
[Chip Design Intern]
[2022.06—2023.06] 
\begin{itemize}
    \item The main content of the internship is to \textbf{design MCU for managing accelerator in 5G baseband chip}. As the main project leader
        I studied the architecture design of the 5G baseband chip analysis its issue. 
        Then I investigate some open source RISC-V processor design as reference as well as RISC-V ISA and RISC-V Toolchain. 
    \item Based on the reserch, we developed the five-stage pipeline architecture of the MCU. I am responsible for the ID, EXE and MEM Stage of the MCU and the verificaton of the MCU core.
\end{itemize}

\ResumeItem{Shanghai Processor Innovation Center}
[Open Source IP Development]
[2023.07—util now] 
The main work of the internship is to provide a fully verified IP component library for \textbf{YSYX project}, and developed an IP design and verification platform;
At present, I am mainly responsible for the design and verification of the VGA module. 
Before tapout, I plan to put the VGA module on the FPGA board for verification to ensure that the IP can be used normally after tapout.
\section{Project Experience}
\ProjectUrl{https://github.com/timemeansalot/FAST_INTR_CPU}{RISC-V Fast Interrupt MCU}
\begin{itemize}
    \item \textbf{Project Requirements}: 
    There are many accelerators in the existing 5G baseband chips. 
    We use Andes's RISC-V Core to schedule the accelerators, but we found that it took hundreds for the Core to response the accelerator's interrupt. 
    So we want to design a RISC-V MCU specially used for baseband chip accelerator scheduling, \textbf{meet the needs of fast response and can be autonomously controlled}.
    \item \textbf{Work Content}: 
    \begin{enumerate}
        \item Analyzing the existing accelerator scheduling problems, the disadvantages are: 1. After the interrupt response is sent, it takes a long time for the software to save the context; 2. Interrupt nesting is not supported, resulting in slow scheduling after multiple accelerators respond.\\
On this basis, the solution of \textbf{hardware context saving} and \textbf{interrupting tail chain} is proposed.
        \item Design MCU architecture: investigated the open source RISC-V processor cores, such as Hummingbird E203, NutShell, Xiangshan, Xuantie, etc;
              The \textit{sequential single-issue five-stage pipeline} architecture is proposed and support the RISC-V 32IMC instruction set; \\
              I am mainly responsible for the design and implementation of \textbf{Fetch, Decode and Execute Stage}, enable instruction pre-taking, branch prediction, compressed instruction alignment, and tight coupling memory.
        \item Build the MCU Core verification platform: the \textit{difftest} verification framework is introduced, 
              Spike is used as the Golden Model to verify the functional correctness of the MCU, and the verification framework is adapted for the self-developed MCU; at present, all riscv-tests test sets have been passed; and the MCU will be further tested on the FPGA
    \end{enumerate}
    \item \textbf{Project Gain}: deep understanding of RISC-V instruction set and open source tool chain; deeper understanding of processor architecture, interrupt, and program execution;
          Programming ability and processor core verification ability; communication, document writing and project management ability.
\end{itemize}

\ProjectUrl{https://github.com/oscc-ysyx-ip-project/ysyx-ip-vga}{Open Source IP Development\footnote{The project name can reach the project homepage.}}

\begin{itemize}
    \item \textbf{Project Requirements}: My main work is to develop VGA modules for the YSYX project board, and provides well-verified VGA modules. 
          The VGA resolutions to be supported are 800x600, 640x480, 480x272, 320x240.
    \item \textbf{Work Content}: Design the specifications, interface and detailed design scheme of the VGA module, 
          as well as the scheme of VGA interaction with the processor core and SDRAM; 
          build the development and verification framework of the IP module, and determine the test point scheme and test vector. 
          At present, the project is still in progress. 
    \item \textbf{Project Gain}:I have learned about VGA, embedded systems and buses; I have deepened my understanding of UVM verification and open source.
\end{itemize}

\section{Personal Summary}
I have rich project and teamwork experience, professional background knowledge, English reading and writing skills. I pursuit of cutting-edge technology, and like to share technology blogs.
\end{document}
