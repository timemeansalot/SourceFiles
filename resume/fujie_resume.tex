% !TeX TS-program = xelatex

\documentclass{resume}
\ResumeName{付杰}

\begin{document}

\ResumeContacts{
  (+86) 151-6213-3916,%
  \ResumeUrl{mailto:timemeansalot@gmail.com}{timemeansalot@gmail.com},%
  \ResumeUrl{https://timemeansalot.github.io/}{付杰的个人博客}
  \footnote{下划线内容包含超链接。},%
  \ResumeUrl{https://github.com/timemeansalot}{Github主页}%
}

\ResumeTitle

\section{教育经历}
\ResumeItem
{上海科技大学与中科院计算所联合培养}
[\textnormal{计算机科学与技术}  学术型硕士研究生]
[2021.09—2024.06]
% [2021.09—2024.06(预计)]

\textbf{GPA: 3.02/4.0},主要研究方向为\textbf{芯片设计、计算机体系结构},
\textbf{2024年应届生},校级奖学金(2次)

\ResumeItem
[中国矿业大学|本科生]
{中国矿业大学}
[\textnormal{计算机科学与技术,计算机学院|} 工学学士]
[2016.09—2020.06]
\textbf{GPA: 3.4/4.0},校级奖学金(1次),院级奖学金(1次),优秀新生奖学金(1次)。

\section{技术能力}
\begin{itemize}
  \item \textbf{语言}: 编程不受特定语言限制。
      常用 Verilog, Python,C; 熟悉C++, Chisel;了解\GrayText{Scala, LaTeX}。
      \footnote{与求职岗位无关的技能省略或用灰色表示。}
  \item \textbf{工作流}: Linux, Makefile, (Neo)Vim, Git, Tmux.
  \item \textbf{其他}: 熟悉RISC-V指令集、Toolchain,了解开源的RISC-V处理器项目。
\end{itemize}

\section{实习经历}

\ResumeItem{中国科学院计算技术研究所}
[芯片设计实习生]
[2022.06—2023.06] 
\begin{itemize}
    \item 实习的主要内容是为\textbf{5G基带芯片中的加速器调度问题设计MCU},
        作为项目主要负责人,\textbf{研究了RISC-V 32IMC指令集及RISC-V Toolchain};
        研究了基带芯片的架构设计、以及开源的RISC-V处理器设计。
    \item 在此基础上\textbf{设计了MCU的五级流水线架构,编写了MCU主要功能部件的RTL代码}。
\end{itemize}

\ResumeItem{上海处理器创新中心}
[\textnormal{开源IP开发}]
[2023.07—至今] 
实习的主要工作是为\textbf{一生一芯项目}开发开源IP,工作重心主要聚焦在数字外设IP核,
后期将会围绕“一生一芯”处理器板卡打造一套开源且经过流片验证的外设组件库,
通过不断迭代的方式逐步优化各个IP的功能和性能,最终实现对常用IP组件的全覆盖,
从而构建一套完整且稳定的开源SoC生态系统,供个人和企业免费使用。

\ResumeItem{深圳无限数科技有限公司}
[架构模型验证]
[2021.01—2021.06] 
\begin{itemize}
  \item 所在项目负责设计一款路由芯片,
      芯片的架构设计首先由C语言编写C模型。
      我主要的工作是\textbf{负责根据项目的设计文档验证C模型的功能是否正确、
      找出C模型的Bug、修复Bug}。测试的时候使用到了Google Test框架。
\end{itemize}

\section{项目经历}
\ResumeItem{RISC-V低功耗快速响应MCU}
\begin{itemize}
    \item 该MCU是基于RISC-V的顺序单发射五级流水线处理器、支持RV-32IMC指令集,其架构设计参考了蜂鸟、果壳和经典五级流水线。
    \item 在\textit{IF Stage}设计了指令预取跟指令FIFO用于处理指令对齐问题;在\textit{ID Stage}设计了译码器、分支预测器、压缩指令扩展、立即数拓展单元;
        在\textit{EXE Stage}支持RV-32IM相关的计算、并且会对分支指令的预测进行验证;在\textit{MEM Stage}和IF Stage,设计了紧耦合存储器来避免Cache Miss
        导致的不确定延时;\textit{WB Stage}会根据指令类型选择数据源写回到通用寄存器组;此外设计了\textit{Hazard Unit}来控制流水线的刷新(flush)和暂停(stall)。
    \item 项目目前使用Verilog语言,目前处理器除了CSR单元外,其余部分代码已经编写完成,处于验证阶段(采用iverilog跟gtkwave);
    \item 搭建MCU Core验证平台,主要工作是搭建Difftest框架,编译riscv-tests源码加载到Difftest,
        将MCU Core运行的结果同Spike进行比较,从而验证MCU Core的功能正确性。
\end{itemize}
\ResumeItem{开源IP开发}
目前主要负责为一生一芯项目组设计VGA开源IP模块,目前该项目还在进行当中,其主要工作包括:

\begin{itemize}
    \item 调研VGA文档、开源项目,在此基础上撰写设计文档、开会论证方案
    \item 根据设计方案编写RTL代码
    \item 搭建IP验证平台,设计测试用例完成IP验证工作
\end{itemize}

\section{个人总结}
\begin{itemize}
    \item 本人乐观开朗、在校成绩优异、自驱能力强、当过助教,具有良好的沟通能力和团队合作精神。
    \item 可以使用英语进行工作交流(六级成绩551),平时有阅读英文论文和口语练习的习惯。
    \item 对Linux下工具熟悉,日常使用tmux, nvim, git;能编写shell, Makefile等
    \item 热爱开源,喜欢分享自己的技术博客,持续关注处理器技术、体系结构技术、AI技术发展。
\end{itemize}

\end{document}
