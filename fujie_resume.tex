% !TeX TS-program = xelatex

\documentclass{resume}
\ResumeName{付杰}

\begin{document}

\ResumeContacts{
  (+86) 151-6213-3916,%
  \ResumeUrl{mailto:timemeansalot@gmail.com}{timemeansalot@gmail.com},%
  \ResumeUrl{https://timemeansalot.github.io/}{付杰的个人博客}
  \footnote{下划线内容包含超链接。},%
  \ResumeUrl{https://github.com/timemeansalot}{Github主页}%
}

\ResumeTitle

\section{教育经历}
\ResumeItem
{上海科技大学与中国科学院大学联合培养}
[\textnormal{计算机科学与技术}  学术型硕士研究生]
[2021.09—2024.06]
% [2021.09—2024.06(预计)]

\textbf{GPA: 3.02/4.0},主要研究方向为\textbf{芯片设计、计算机体系结构},
\textbf{2024年应届生},校级奖学金(3次)

\ResumeItem
[中国矿业大学|本科生]
{中国矿业大学}
[\textnormal{计算机科学与技术,计算机学院|} 工学学士]
[2016.09—2020.06]
\textbf{GPA: 3.4/4.0},校级奖学金(1次),院级奖学金(1次),优秀新生奖学金(1次)。

\section{技术能力}
\begin{itemize}
  \item \textbf{语言}: 编程不受特定语言限制。
      常用 Verilog, Python,C; 熟悉C++, Chisel;了解\GrayText{Scala, LaTeX}。
      \footnote{与求职岗位无关的技能省略或用灰色表示。}
  \item \textbf{工作流}: Linux, Makefile, (Neo)Vim, Git, Tmux.
  \item \textbf{其他}: 熟悉RISC-V指令集、Toolchain,了解开源的RISC-V处理器项目。
\end{itemize}

\section{实习经历}

\ResumeItem{中国科学院计算技术研究所}
[芯片设计实习生]
[2022.06—至今] 
\begin{itemize}
    \item 实习的主要内容是为\textbf{5G基带芯片中的加速器调度问题设计MCU},
        作为项目主要负责人,\textbf{研究了RISC-V 32IMC指令集及RISC-V Toolchain};
        研究了基带芯片的架构设计、以及开源的RISC-V处理器设计。
    \item 在此基础上\textbf{设计了MCU的五级流水线架构,编写了MCU主要功能部件的RTL代码}。
\end{itemize}
\ResumeItem{深圳无限数科技有限公司}
[架构模型验证]
[2021.01—2021.06] 
\begin{itemize}
  \item 所在项目负责设计一款路由芯片,
      芯片的架构设计首先由C语言编写C模型。
      我主要的工作是\textbf{负责根据项目的设计文档验证C模型的功能是否正确、
      找出C模型的Bug、修复Bug}。测试的时候使用到了Google Test框架。
\end{itemize}

\section{项目经历}

\ResumeItem{RISC-V低功耗快速响应MCU}
\begin{itemize}
    \item 根据加速器调度特点以及快速响应的需求,
        基于RISC-V 32IMC指令集设计了MCU的整体架构。
    \item IF Stage设计了5*16的指令FIFO用于处理压缩指令的读取对齐问题,
        存储部分采用ITCM作为存储体以避免Cache Miss导致的时延、保证实时性。
    \item ID Stage设计了:压缩指令译码器、译码器、静态分支预测器、扩展单元,分别用于实现
           判断压缩指令并且将压缩指令恢复成32bits的指令;
           对32bits指令进行译码,生成控制信号;
           对分支指令进行静态分支预测,降低CPI;
           将指令中的立即数扩展为32bits,用于后续的计算。
    \item EXE Stage在设计的时候尽可能地复用了加法器,以降低MCU的面积和功耗。
    \item MEM Stage对访存指令设计了与Data-Memory的交互接口, Data-Memory采用DTCM。
    \item WB Stage是纯组合逻辑电路,用于选择合适的数据源写入到寄存器组中。
    \item Hazard Unit: 用于判断指令之间的依赖问题,
        一般的数据依赖通过bypass到ID Stage来消除;
        不能通过bypass消除的数据依赖,则通过stall流水线来消除。
\end{itemize}

\ResumeItem{\textbf{RISC-V SoC}设计与实现\footnote{目前还在进行中的项目}}
[ 适配\textnormal{MCU}]
为MCU核搭建配套的SoC
\begin{itemize}
    \item 仿真验证:编译环境与库函数的建立, 
        实现了SoC子模块自动化的完备验证方法、对SoC性能整体评估的方法。
    \item 调试: 设计了MCU对外的调试接口,采用JTAG进行调试、UART打印调试信息。
        接口:设计了MCU的内部总线,用以MCU和加速器之间的信息交互。
    \item 协处理器:设计协处理器接口控制电路,接入MCU内核,设计相应的协处理器指令并建立编译环境,
        使得MCU可以通过自定义扩展指令访问协处理器
\end{itemize}

\ResumeItem{一生一芯处理器}
[ 基于\textnormal{Chisel}敏捷开发]
\begin{itemize}
  \item 基于Chisel语言敏捷开发一款2级流水线的低功耗嵌入式处理器核。 
\end{itemize}

\section{个人总结}
\begin{itemize}
    \item 本人乐观开朗、在校成绩优异、自驱能力强,具有良好的沟通能力和团队合作精神。
    \item 可以使用英语进行工作交流(六级成绩551),平时有阅读英文论文和口语练习的习惯。
    \item 有6年Linux使用经验;有较为丰富的IC开发设计经验。
    \item 善于技术写作,持续关注处理器技术、体系结构技术发展。
\end{itemize}


\end{document}
